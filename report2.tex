\documentclass[12pt,a4paper]{article}

\usepackage[cm]{fullpage}
\usepackage{amsthm}
\usepackage{amsmath}
\usepackage{amsfonts}
\usepackage{amssymb}
\usepackage{xspace}
\usepackage[english]{babel}
\usepackage{fancyhdr}
\usepackage{titling}
\usepackage{listings}
\renewcommand{\thesection}{Exercise \Alph{section}:}

%%%%%%%%%%%%%%%%%%%%%%%%%%%%%%%%%%%%%%%%%%
% This part needs customization from you %
%%%%%%%%%%%%%%%%%%%%%%%%%%%%%%%%%%%%%%%%%%

% please enter exercise sheet number, your name, matriculation, and a
% list of students which you worked together with here
\newcommand{\name}{Jakob Fischer}
\newcommand{\matriculation}{01427178}
\newcommand{\fellows}{None}

%%%%%%%%%%%%%%%%%%%%%%%%%%%%%%%%%%%%%%%%%%
%           End of customization         %
%%%%%%%%%%%%%%%%%%%%%%%%%%%%%%%%%%%%%%%%%%

\author{\name\ (\matriculation)}

\newcommand{\projnumber}{2}
\newcommand{\Title}{DAO Down}
\setlength{\headheight}{15.2pt}
\setlength{\headsep}{20pt}
\setlength{\textheight}{680pt}
\pagestyle{fancy}
\fancyhf{}
\fancyhead[L]{Cryptocurrencies - Project \projnumber\ - DAO Down}
\fancyhead[C]{}
\fancyhead[R]{\name}
\renewcommand{\headrulewidth}{0.4pt}
\fancyfoot[C]{\thepage}


\begin{document}
\thispagestyle{empty}
\noindent\framebox[\linewidth]{%
 \begin{minipage}{\linewidth}%
 \hspace*{5pt} \textbf{Cryptocurrencies (WS2017/18)} \hfill Prof.~Matteo Maffei \hspace*{5pt}\\

 \begin{center}
  {\bf\Large Project \projnumber~-- \Title}
 \end{center}

 \vspace*{5pt}\hspace*{5pt} \hfill TU Wien \hspace*{5pt}
\end{minipage}%
}
\vspace{0.5cm}
%%%%%%%%%%%%%%%%%%%%%%%%%%%%%%%%%%%%%%%%%%%%%%%

\begin{center}
  Submission by \textbf{\theauthor}\\[1cm]

  During the development of the project solutions, I have discussed
  problems, solutions, and other questions with the following other
  students:\\
  \fellows %please fill the information above
\end{center}

\section*{Attack report}

For the attack, two contracts were created. The abstract contract "Dao" only provides the used method signatures of EDAO, to make needed calls to EDAO easier.
"Contract" is the contract which will attack EDAO.

\subsection*{Contract variables}
The contract has three constants:
\begin{itemize}
\item \textbf{owner} The address of the owner of the contract. Not really used.
\item \textbf{target} The address of the target contract, cast to Dao.
\item \textbf{drainPerCall} The amount of wei which should be withdrawn everytime the contract's fallback function gets called.
As a result, the target contract must have at least drainPerCall funds for this contract. Otherwise, the if-condition in EDAO.withdraw could not be passed.
\end{itemize}

\subsection*{Events}
The contract contains 2 Events to make debugging easier:
\begin{itemize}
\item \textbf{BeforeDrain(uint256)} Will be invoked everytime fallback function is
called and contains the current balance of the target contract, which at the end should be zero.
\item \textbf{DrainFinished(uint256)} Will be invoked when the target has no balance left
and the contract stops calling the withdraw function of the target contract. It logs
the balance of the current contract, which at the end should be the starting balance of the target contract (before \textit{start\_attack} was called).
\end{itemize}

\subsection*{Functions}
The contract contains some helper functions for debugging, but which are not attack relevant. They are not mentioned here.
\begin{itemize}
\item \textbf{start\_attack} Initiates the first withdraw call to the target contract.
\item \textbf{fallback function} As long as the target contract has balance left, the function will call the EDAO.withdraw function.
Usually, 1 ETH (as defined in drainPerCall) will be withdrawn every time.
If the target contract has less than that value left, it will withdraw the remaining balance.
Otherwise we could not drain all ETH from EDAO, because of the second part of the if-condition in EDAO.withdraw.
\end{itemize}

\subsection*{Attack explanation}
The attack works, because of the order of the payout/decrease funds instructions in the withdraw function of EDAO.

To payout the funds, the function would first call the fallback function of the fund receiver (and ETH will be transfered from EDAO to attacker).
Only after that call would return, the funds variable would be adjusted, to decrease the remaining funds of the receiver.

However, the attacking contract, in it's fallback function, would call the withdraw function again (and the fallback-call of EDAO would not yet return).
Since the funds haven't been adjusted already, the attacker would have again enough funds to withdraw, and it's fallback function will be called again by EDAO.

When there is no balance left, the attacker can stop the recursion, by just returning.
Only after that point the fallback-calls of the withdraw function would return, and the funds can be decreased
(and underflowed, because funds are unsigned. As I write this, it came to my mind that maybe this could have been another way to drain all funds.).

To pass the if-condition at the beginning of the withdraw function, the attacking contract must have received enough funds first.
I funded always 1 ETH, and so drainPerCall would be set to the same amount.

Used Sources:
https://allquantor.at/blockchainbib/pdf/atzei2016survey.pdf

\subsection*{Difficulties}
One diffulty I encountered, was how I could call other's contracts functions. I tried it with "call" first, but I had some problems with it.
In the end, I used an abstract contract for calls to EDAO, which seemed the most convenient way.

During the development of the contract, I came across Remix (https://remix.ethereum.org/), which made testing the contract a lot easier.
Otherwise, it would have been very tedious to compile and deploy the contract for each small fix.
However, I couldn't get my abstract contract compiled in Remix so I had to add some basic implementation in it.

Another difficulty was the sending of transactions in geth - there seemed to be more than one method of it, all working a little bit different.
For one method, it wasn't at first clear to me, where I had to supply \textit{from}, \textit{value} and where I had to supply the functions arguments.

\pagebreak

\subsection*{Contract code}
Not sure if I should put this here, but I do it anyway.

\lstinputlisting{contract.sol}

\end{document}
